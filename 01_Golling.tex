% Created 2021-09-15 Mi 13:36
% Intended LaTeX compiler: pdflatex
\documentclass[pdftex,a4paper,12pt,bibliography=totoc,draft]{scrartcl}
                              \usepackage[top=2.5cm, bottom=2.5cm, left=4.0cm, right=3.0cm]{geometry}
\usepackage{nicefrac}
\usepackage{amsmath}
%\usepackage{ucs}
\usepackage[germanb]{babel}
\usepackage[utf8]{inputenc}
\usepackage[T1]{fontenc}
\usepackage{textcomp}
\RequirePackage[ngerman=ngerman-x-latest]{hyphsubst}

\usepackage[style=authoryear-ibid,backend=biber,dashed=false,isbn=false]{biblatex}
\DeclareLanguageMapping{ngerman}{ngerman-apa}

\makeatletter
\renewbibmacro*{bbx:editor}[1]{%
\ifthenelse{\ifuseeditor\AND\NOT\ifnameundef{editor}}
{\ifthenelse{\iffieldequals{fullhash}{\bbx@lasthash}\AND
\NOT\iffirstonpage\AND
\(\NOT\boolean{bbx@inset}\OR
\iffieldequalstr{entrysetcount}{1}\)}
{\bibnamedash}
{\printnames{editor}%
\setunit{\addspace}% GEÄNDERT
\usebibmacro{bbx:savehash}}%
\printtext[parens]{\usebibmacro{#1}}% GEÄNDERT
\clearname{editor}%
\setunit{\addspace}}%
{\global\undef\bbx@lasthash
\usebibmacro{labeltitle}%
\setunit*{\addspace}}%
\usebibmacro{date+extradate}}
\makeatother

\DeclareBibliographyDriver{incollection}{%
\usebibmacro{bibindex}%
\usebibmacro{begentry}%
\usebibmacro{author/translator+others}%
\setunit{\labelnamepunct}\newblock
\usebibmacro{title}%
\newunit
\printlist{language}%
\newunit\newblock
\usebibmacro{byauthor}%
\newunit\newblock
\usebibmacro{in:}%
\begingroup% NEU
\renewbibmacro*{date+extradate}{}% NEU
\usebibmacro{editor+others}% NEU
\newunit{\addcolon\addspace}\newblock% NEU
\endgroup% NEU
\usebibmacro{maintitle+booktitle}%
\newunit\newblock
%  \usebibmacro{byeditor+others}%
%  \newunit\newblock
\newunit\newblock
\usebibmacro{chapter+pages}%
\printfield{edition}%
\newunit
\iffieldundef{maintitle}
{\printfield{volume}%
\printfield{part}}
{}%
\newunit
\printfield{volumes}%
\newunit\newblock
\usebibmacro{series+number}%
\newunit\newblock
\printfield{note}%
\newunit\newblock
\usebibmacro{publisher+location+date}%
%\newunit\newblock
%\usebibmacro{chapter+pages}%
\newunit\newblock
\iftoggle{bbx:isbn}
{\printfield{isbn}}
{}%
\newunit\newblock
\usebibmacro{doi+eprint+url}%
\newunit\newblock
\usebibmacro{addendum+pubstate}%
\newunit\newblock
\usebibmacro{pageref}%
\usebibmacro{finentry}}

% \DeclareBibliographyDriver{article}{
%   \usebibmacro{bibindex}%
%   \usebibmacro{begentry}%
%   \usebibmacro{author/translator+others}%
%   \setunit{\labelnamepunct}\newblock
%   \usebibmacro{title}%
%   \newunit
%   \printlist{language}%
%   \newunit\newblock
%   \usebibmacro{byauthor}%
%   \newunit\newblock
%   \usebibmacro{byeditor+others}%
%   \newunit\newblock
%   \printfield{version}%
%   \newunit\newblock
% %  \usebibmacro{in:}%   SO KLAPPT DAS!
%   \usebibmacro{journal+issuetitle}%
%   \newunit\newblock
%   \printfield{note}%
%   \setunit{\bibpagespunct}%
%   \printfield{pages}
%   \newunit\newblock
%   \printfield{issn}%
%   \newunit\newblock
%   \printfield{doi}%
%   \newunit\newblock
%   \usebibmacro{eprint}
%   \newunit\newblock
%   \usebibmacro{url+urldate}%
%   \newunit\newblock
%   \printfield{addendum}%
%   \newunit\newblock
%   \usebibmacro{pageref}%
%   \usebibmacro{finentry}
% }

\DefineBibliographyStrings{german}{%
byeditor = {Hrsg\adddot},%
byeditor = {Hrsg\adddot},%
andothers={et \addabbrvspace al\adddot},
}

%\DeclareNameAlias{sortname}{last-first}
%\DeclareNameAlias{default}{last-first}
\DeclareNameAlias[incollection]{editor}{default}
%\DeclareFieldFormat{namelast}{\mkbibacro{#1}}
\DeclareFieldFormat[incollection]{title}{\mkbibitalic{#1}}
\DeclareFieldFormat[incollection]{booktitle}{#1}
\DeclareFieldFormat[incollection]{pages}{\parentext{#1}}
\DeclareFieldFormat[incollection]{editor}{#1\addcolon}
\DeclareFieldFormat[article]{title}{\mkbibitalic{#1}}
\DeclareFieldFormat[article]{journaltitle}{#1}
\DeclareFieldFormat[article]{urldate}{\brackettext{#1}}
\DeclareFieldFormat[article]{note}{#1\addcolon\addspace}
\DeclareFieldFormat[collection]{urldate}{\brackettext{#1}}
\DeclareFieldFormat[collection]{note}{#1\addcolon\addspace}
\DeclareFieldFormat[misc]{urldate}{\brackettext{#1}}
\DeclareFieldFormat[misc]{note}{#1\addcolon\addspace}


\AtBeginBibliography{%
\renewcommand{\nametitledelim}{\addcolon\space}
\renewcommand{\finalnamedelim}{\addcomma\space}

}
%\renewcommand{\mkbibnamefamily}[1]{\textsc{#1}}
\renewcommand{\labelnamepunct}{\addcolon\addspace}
\renewcommand{\nameyeardelim}{\addcomma\space}
\renewcommand{\subtitlepunct}{\adddot\space}




%\renewcommand{\nametitledelim}{\addcolon\addspace}
%\addbibresource{Literatur.bib} %% Einbinden der bib-Datei
\bibliography{../Bibliography/Literatur}
\providecommand{\apashortdash}{-}


%\makeatletter
%\renewcommand\@biblabel[1]{}
%\makeatother
\usepackage[babel,german=quotes]{csquotes}
\usepackage{url}
\urlstyle{rm}
\usepackage[pdftex]{graphicx}
\usepackage{hyperref}
\usepackage{cjhebrew}
\renewcommand{\figurename}{Abbildung}
\usepackage{pdfpages}
\renewcommand{\familydefault}{\rmdefault}
\usepackage{times}
\addtokomafont{sectioning}{\rmfamily}
\usepackage{setspace}
%\renewcommand{\familydefault}{\sfdefault}
%\usepackage{helvet}
%\usepackage{lmodern}
\usepackage{booktabs}
\usepackage{ragged2e}
\RequirePackage{processkv}
\usepackage{parcolumns}
\usepackage{blindtext}
\setcounter{tocdepth}{3}
\setcounter{secnumdepth}{3}
\usepackage{xpatch}



%Kopfzeile
\usepackage[draft=false,automark,headsepline,plainheadsepline]{scrlayer-scrpage}
%\KOMAoptions{onpsinit=\linespread{1}\selectfont
\pagestyle{scrheadings}
\clearmainofpairofpagestyles
\clearplainofpairofpagestyles
\ihead{\headmark}
\ohead{\pagemark}
\automark{section}
\onehalfspacing
\usetheme{default}
\author{Matthias Fuchs}
\date{\today}
\title{[SW20] Für einen treuen Freund gibt es keinen Preis}
\begin{document}

\maketitle
\begin{frame}{Outline}
\tableofcontents
\end{frame}

\begin{frame}[label={sec:orgf12252a}]{Kompetenzbeschreibung}
Menschen und ihre Lebensorientierungen: Beziehung verantwortungsvoll gestalten können – zu sich selbst, zu anderen, zur Schöpfung
Die Schülerinnen und Schüler können sich in ihrer gottgeschenkten Einzigartigkeit wahrnehmen und wissen um die Bedeutung von (Selbst-) Vertrauen für ein gelingendes (Zusammen-)Leben.
\end{frame}

\begin{frame}[label={sec:orgc05fdb9}]{AB / UH}
Ich und die amderen: Freundschaft
\end{frame}

\begin{frame}[label={sec:org0a2dbec}]{Wiederholung}
Was bedeutet dir Freundschaft?
\end{frame}

\begin{frame}[label={sec:org5dbdc9e}]{Erarbeitung}
Bsp aus der Bibel zum Thema "Freundschaft":
\begin{itemize}
\item Verse aus den Büchern Sprüchwörter, Jesus Sirach, Weisheit.
\item David und Jonathan (2Sam 1,25); Rut und Noomi (Sir 6; 22,19-26;37,1-6)
\item "Freund des Königs" ist ein festes Amt (2Sam 15,37)
\item Abraham und Mose werden "Freunde Gottes" gennannt (2Chr 20,7; Ex 33,11)
\item Jesus nennt seine Jünger Freunde (Lk 12,4; Joh 15,13)
\item Lazarus, Martha und Maria sind Freunde Jesu (Joh 11)
\item Johannes der Täufer als Bräutigam und Freund (Joh 3,29)
\item Jesus und Johannes ("Jünger, den Jesus liebte")
\item Jesus wird von seinem Freund Judas verraten und ausgeliefert (Mt 26,50)
\end{itemize}

\href{https://www.youtube.com/watch?v=3hR64GCyOg0}{Woran erkenne ich eine ungesunde Beziehung - Johannes Hartl - YouTube}
Das Gleiche wollen, das Gleiche nicht wollen - daraus entsteht Beziehung. Man kann aber auch gemeinsam das Schlechte wollen (Herodes / Pilatus).
Das Gleiche wollen, heißt noch nicht, das es eine gute Beziehung ist.

\begin{itemize}
\item Woran erkenne ich, ob eine Beziehung gut ist?
\item Mut haben es anderen zu sagen, dass deren Beziehung nicht gut ist
\item Wo sind mögliche Baustellen in meiner Beziehung und wie kann ich sie "beheben"
\end{itemize}

Es gibt eigentlich keine gesunde Beziehung, weil jeder Mensch irgendwo einen Knacks hat. Wir lernen Beziehung von Menschen, die selber einen Knacks haben. Nur einer lebte vollkommenene Beziehungen: Jesus. "Du bist mein geliebter Sohn." Jesus ist sehr unabhängig in seiner Bez. zu Menschen. Er kann selber stehen. Jesus hat auch keine Angst vor Nähe, er lässt sich berühren, es gibt nichts Kühles. Aber er bleibt frei. "Nein, ich muss weiter gehen, ich komme nicht in euer Dorf." Freiheit / Nähe. Er kann sogar erschreckend klar sein. Wahrheit.
\end{frame}



\begin{frame}[label={sec:orge45485a}]{Hefteintrag}
\end{frame}
\end{document}